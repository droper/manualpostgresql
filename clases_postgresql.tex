%%%%%%%%%%%%%%%%%%%%%%%%%%%%%%%%%%%%%%%%%%%%%%%%%%%%%%%%%%%%%%%%%%%%%%%%%%%
%
% Clases de Postgresql
%
%%%%%%%%%%%%%%%%%%%%%%%%%%%%%%%%%%%%%%%%%%%%%%%%%%%%%%%%%%%%%%%%%%%%%%%%%%%

\documentclass[12pt,a4paper,oneside, openany]{book}
\textheight=26cm
\textwidth=18cm
\topmargin=-2cm
\oddsidemargin=-1cm
\parindent=0mm

\usepackage[utf8x]{inputenc}
\usepackage[spanish]{babel}
\usepackage{ucs}
\usepackage{amsmath,amsfonts,amssymb}
\usepackage{graphicx}
\usepackage{anyfontsize}
\usepackage{float}
\usepackage{textcomp}
%Paquete verbments para obtener bonito código fuente
\usepackage{verbments}

\fvset{frame=single,framerule=1pt}
\plset{language=sql,texcl=true,style=vs,%
captionfont=\sffamily\color{white}}

\setcounter{chapter}{0}

\pagestyle{plain}


%--------------------------------------------------------------------------
%titulo
\title{ 
 \begin{center}
  {Introducción a PostgreSQL}\\
 \end{center}
}

\author{Ing. Pedro Muñoz del Rio\\
        pmunoz@gmail.com}

\date{Lima, Perú}

\begin{document}

\renewcommand{\baselinestretch}{1.5} %espacio entre lineas, 2 es doble espacio
\renewcommand{\contentsname}{Índice}
\renewcommand{\listfigurename}{Lista de Gr\'aficos}
\renewcommand{\chaptername}{Cap\'itulo}
\renewcommand{\bibname}{Fuentes de Informaci\'on}
\renewcommand{\figurename}{Gr\'afico}
\renewcommand{\tablename}{Tabla}

 \maketitle %despliega el titulo
 
  \frontmatter  %Secciones del documento que van con numeros romanos en el indice 
 
   \chapter{Introducción}

PostgreSQL es un gestor de bases de datos relacionales que empezó como un proyecto de la universidad de Berkeley en California y su historia data desde los comienzos de las bases relacionales.\\

Posee funcionalidades de corte empresarial como funciones para crear vistas, funciones agregadas para aplicar sobre las vistas entre otras. Tiene la capacidad de aceptar grandes cantidades de datos, como tablas con decenas de millones de registros sin mayor problema.\\

Además de un gestor de bases de datos, PostgreSQL es una plataforma para aplicaciones y permite escribir procedimientos almacenados y funciones en diversos lenguajes como SQL, Python y Java entre otros, agregando los módulos necesarios en el caso de los últimos.\\

Conectarse a un web service a través de Python, usar funciones estadísticas de R y consultar los resultados con SQL es posible mediante PostgreSQL.\\

Además, se pueden definir tipos propios de datos, instalar fácilmente extensiones mediante una sola instrucción SQL y administrarlas con sencillez.\\

Y como si fuera poco, Postgresql es multiplataforma, por lo cuál puede ser implementado tanto en Linux, BSD como Windows e incluso Mac OS, con binarios disponibles si no se desea compilar postgresql.\\

Los lenguajes de programación más populares tienen librerías que les permiten comunicarse con Postgresql y hacer consultas, entre los más conocidos PHP, Python, Ruby, Java, .Net, Perl entre otros.\\

Hoy en día, PostgreSQL es una de las alternativas más relevantes en el campo de las Bases de Datos relacionales, de código abierto y utilizada por empresas de todo tamaño a nivel mundial.\\ %Resumen y Abstract
  
 \tableofcontents %despliega el índice
 \listoffigures   %Tabla de figuras
      
 \mainmatter  %Cuerpo del documento  

 %Cuerpo del documento 

 \chapter{Arquitectura del Sistema}
    \input arquitectura_postgresql.tex

 \chapter{Administración}
    \input administracion_postgresql.tex

 \chapter{Transacciones y concurrencia}
    \input transacciones_postgresql.tex
    
 \chapter{Ajustes de rendimiento}
 	\input ajustes_postgresql.tex
 	
 \chapter{Ajustes de rendimiento II: Optimización de las Bases de datos}
 	\input ajustes_postgresql_querys.tex
 	
 \chapter{Alta disponibilidad y replicación}
 	\input alta_disponibilidad_postgresql.tex
 	
 \chapter{Particionado de tablas}
 	\input particionado_tablas_postgresql.tex 	
 	
 \chapter{Pooling de conexiones}
	\input pooling_conexiones_postgresql.tex 	
 	
 \chapter{Lenguajes Procedurales}
 	\input lenguajes_procedurales_postgresql.tex 	 	 	
 
 \chapter{Extensiones y contrib}
 	\input extensiones_contrib.tex  		 	 	

   \bibliographystyle{miestilo}
   \bibliography{bibliografia}

\end{document}