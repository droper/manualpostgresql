Las extensiones y los módulos contribs son Add-ons que se pueden instalar en una base de datos para  extender su funcionalidad más allá de lo que ofrece PostgreSQL. \\

Antes de PostgreSQL 9.1 los add-ons se llamaban contribs, a partir de esa versión, los add-ons son instalados con facilidad mediante el nuevo modelo de extensiones, término que ha reemplazado a contrib.\\

Las extensiones son instaladas por separado en cada base de datos. Se puede tener una base de datos con una extensión y las demás no tenerla. Si se desea que todas las bases de datos tengan un esquema, este se debe instalar en la base de datos \textit{template1} para que al crearse una nueva, tenga a su vez la extensión deseada.\\

Para ver las extensiones instaladas se ejecuta:\\

\textit{SELECT * FROM pg\_available\_extensions;}\\

Para tener los detalles de una extensión en particular se utiliza el siguiente comando:\\

\textit{\textbackslash dx+ plpgsql}\\

\section{Instalar Extensiones}

Para instalar una extensión se deben de colocar sus scripts en \textit{share/extension} y sus binarios en \textit{lib} y \textit{bin}. En el caso de las extensiones más comunes, estas ya se encuentran precompiladas y solo falta instalarlas.\\

La forma moderna y sencilla de instalar una extensión es:\\

\emph{CREATE EXTENSION fuzzystrmatch;}\\

Si se desea instalar la extensión en un esquema particular, primero se crea el esquema y luego se instala la extensión:\\

\emph{CREATE EXTENSION fuzzystrmatch SCHEMA my\_extensions;} \\

\section{Extensiones Comunes}

Diversas extensiones vienen empaquetadas con PostgreSQL pero no son instaladas por defecto. Algunas de las más populares son:\\

\begin{itemize}
\item \textbf{Postgis}: Una extensión que coloca a PostgreSQL entre las principales bases de datos geográficas. Está compuesto de más de 800 funciones, tipos e índices espaciales.
\item \textbf{fuzzystrmatch}: Es una extensión ligera para trabajar de forma difusa con cadenas.
\item \textbf{hstore}: Es una extensión que agrega almacenamiento por llave y valor (tipo diccionario) para guardar data pseudo normalizada. Útil para servir de intermedio entre relacional y NoSQL.
\item \textbf{pg\_trgm}: Una extensión utilizada para manejar de forma difusa cadenas de texto.
\item \textbf{dblink}: Módulo que permite hacer consultas a otras bases de datos. Es el único mecanismo para trabajar entre bases de datos con PostgreSQL.
\item \textbf{pgcrypto}: Provee herramientas de cifrado incluyendo PGP. 
\end{itemize}

