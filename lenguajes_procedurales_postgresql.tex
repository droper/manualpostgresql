Como en otras bases de datos, se pueden juntar expresiones SQL y volverlas una unidad. tiene diferentes nombres según el motor utilizado: procedimientos almacenados, módulos, macros, etc. En PostgreSQL se llaman funciones y además de permitir juntar varias expresiones SQL, permiten también utilizar lógica para dirigir el flujo del programa.\\

En PostgreSQL existen diversos lenguajes que se pueden utilizar para escribir funciones y que con frecuencia vienen ya empaquetados: SQL, C, PL/pgSQL, PL/Perl, PL/Python. Se pueden instalar además otrs lenguajes como R, Java, sh e incluso algunos experimentales como Scheme. Para ver una lista de los lenguajes disponibles acceder a:\\

http://www.postgresql.org/docs/current/interactive/external-pl.html\\

\section{Estructura de una función }

Sin importar que lenguaje se escoge para escribir una función, todas comparten una misma estructura:\\

\begin{pyglist}
CREATE OR REPLACE FUNCTION func_name(
     arg1 arg1_datatype)
RETURNS some_type | setof sometype | TABLE (..) AS
$$
BODY of function
$$
LANGUAGE language_of_function
\end{pyglist}

Para listar la lista de lenguajes en la base de datos se ejecuta:\\

\begin{pyglist}
SELECT lanname FROM pg_language;
\end{pyglist}

La definición de una función puede incluir atributos adicionales para optimizar la ejecución y mejorar la seguridad:\\

\begin{itemize}
\item LANGUAGE: Debe ser uno instalado en la base de datos.
\item VOLATILITY: Por defecto es VOLATILE si no es especificado. Puede ser establecido como STABLE, VOLATILE o IMMUTABLE. Estas configuraciones le dan una idea al planificador sobre si los resultados de la función pueden ser colocados en la cache. STABLE significa que la función va a retornar el mismo valor para la misma entrada y la misma consulta. VOLATILE significa que la función puede retornar un valor diferente con cada llamada del sistema. INMUTABLE significa que dada una entrada, la salida siempre va a ser la misma.
\item STRICT: Una función es asumida como no estricta a menos que tenga el atributo STRICT. Una función STRICT siempre retorna NULL si alguna entrada es NULL. 
\item COST: Es una medida relativa de la intensidad de computo. Las funciones SQL y PL/PgSQL tienen un valor por defecto de 100 y las de C de 1. Amayor el valor, se asume que la función es más costosa.
\item ROWS: Es un estimado de cuantos registros va a devolver la función. El formato es ROWS 100.
\item SECURITY DEFINER: Cláusula opcional que implica que la función se ejecuta en el contexto del dueño dela función. Si no se coloca, la función se ejecuta en el contexto del usuario actual.
\end{itemize}

\section{Funciones SQL}

Escribir funciones en SQL es sencillo. Se agrega la estructura y se tiene la función. Debido a que no es un lenguaje propiamente dicho, en SQL no hay estructuras de control. No puede haber más de una sentencia SQL (se pueden utilizar sub consultas).\\

Ejemplo:\\

\begin{pyglist}
CREATE OR REPLACE FUNCTION ins_logs(param_user_name varchar, param_description text)
  RETURNS integer AS
$$ INSERT INTO logs(user_name, description) VALUES($1, $2)
RETURNING log_id; $$
LANGUAGE 'sql' VOLATILE;
\end{pyglist}

Para llamar a la función se ejecuta:\\

\begin{pyglist}
SELECT ins_logs('lhsu', 'this is a test') As new_id;
\end{pyglist}

Para retornar conjuntos de registros existes tres alternativas: RETURNS TABLE, usar parámetros OUT, o retornar un tipo de datos compuesto. \cite{Obe2012}\\

Ejemplo de RETURNS TABLE:\\

\begin{pyglist}
CREATE FUNCTION sel_logs_rt(param_user_name varchar)
RETURNS TABLE (log_id int, user_name varchar(50), description text, log_ts timestamptz) AS
$$
SELECT log_id, user_name, description, log_ts FROM logs WHERE user_name = $1;
$$
LANGUAGE 'sql' STABLE;
\end{pyglist}

Utilizando parámetros OUT:\\

\begin{pyglist}
CREATE FUNCTION sel_logs_out(param_user_name varchar, OUT log_id int
 , OUT user_name varchar, OUT description text, OUT log_ts timestamptz)
RETURNS SETOF record AS
$$
SELECT * FROM logs WHERE user_name = $1;
$$
LANGUAGE 'sql' STABLE;
\end{pyglist}

Usando un tipo de dato compuesto:\\

\begin{pyglist}
CREATE FUNCTION sel_logs_so(param_user_name varchar)
RETURNS SETOF logs AS
$$
SELECT * FROM logs WHERE user_name = $1;
$$
LANGUAGE 'sql' STABLE;
\end{pyglist}



Todas las funciones pueden ser llamadas usando:\\

\begin{pyglist}
SELECT * FROM sel_logs_rt('lhsu');
\end{pyglist}

\section{Funciones PL/pgSQL}

Cuando las necesidades escapan de los límites del SQL puro, la opción más común es PL/pgSQL. Se diferencia de SQL en que se pueden declarar variables utilizando DECLARE, existen instrucciones de control de flujo y el cuerpo de la función debe de estar entre BEGIN y END.\\

Ejemplo:\\

\begin{pyglist}
CREATE FUNCTION sel_logs_rt(param_user_name varchar)
RETURNS TABLE (log_id int, user_name varchar(50), description text,
               log_ts timestamptz) AS
$$
BEGIN
    RETURN QUERY
    SELECT log_id, user_name, description, log_ts
      FROM logs WHERE user_name = param_user_name;
END;
$$
LANGUAGE 'plpgsql' STABLE;
\end{pyglist}




\section{Funciones en PL/Python}

Python es un lenguaje popular con gran cantidad de librerías. En PostgreSQL se pueden escribir funciones utilizando Python. Están soportadas tanto Python 2 como Python 3.\\

Para utilizar Python, se debe instalar el paquete de software que contiene la extensión. en Ubuntu:\\

\emph{sudo apt-get install postgresql-plpython-9.2}\\

Luego se ingresa en la base de datos y se instala la extensión:\\

\begin{pyglist}
CREATE EXTENSION plpython2u;
\end{pyglist}

\subsection{Función python básica}

PostgreSQL convierte automáticamente los tipos de PostgreSQL a tipos de dato Python y viceversa. Desde Pl/Pyton se pueden retornar arreglos e incluso tipos de datos compuestos. Se puede utilizar python para escribir Triggers y crear funciones de agregación.\\

Pl/Python es un lenguaje ``untrusted'' debido a que puede interactuar con el sistema operativo y por lo tanto una función solo puede ser creada por un superusuario. Un usuario puede escribir funciones en PL/Python si es que al crearse se le dio el atributo SECURITY DEFINER. Los lenguajes ``untrusted'' se reconocen debido a que llevan una ``u'' al final de su nombre.\\

Python permite llevar a cabo tareas que no son factible en SQL o PL/pgSQL. En el siguiente ejemplo se muestra como hacer una función que haga una búsqueda de texto en la web de documentación de PostgreSQL.\\

\begin{pyglist}
CREATE OR REPLACE FUNCTION postgresql_help_search(param_search text)
RETURNS text AS
$$
import urllib, re
response = urllib.urlopen \ 
('http://www.postgresql.org/search/?u=%2Fdocs%2Fcurrent%2F&q='+param_search)
raw_html = response.read()
result = raw_html[raw_html.find("<!-- docbot goes here -->"):raw_html.find("<!-- \
pgContentWrap -->") - 1]
result = re.sub('<[^<]+?>', '', result).strip()
return result
$$
LANGUAGE plpython2u SECURITY DEFINER STABLE;
\end{pyglist}

Lo que hace la función es:\\

\begin{enumerate}
\item Importar las librerías a utilizar.
\item Concatenar el término de búsqueda con la dirección web.
\item Leer la respuesta y guardarla en una variable llamada raw\_html.
\item Guardar la parte de raw\_html que empieza con $<$!-- docbot goes here --$>$ y termina antes de $<$!-- pgContentWrap --$>$.
\item Elimina los tags HTML y los espacios en blanco al inicio y final y guarda en result.
\item Retorna el resultado final result.
\end{enumerate}

Llamar a una función Python no es diferente que llamar a funciones en otros lenguajes:\\

\begin{pyglist}
SELECT search_term, left(postgresql_help_search(search_term), 125) 
As result FROM (VALUES
('regexp_match'), ('pg_trgm'), ('tsvector')) As X(search_term);
\end{pyglist}

Otro ejemplo permitirá interactuar con el sistema operativo:\\

\begin{pyglist}
CREATE OR REPLACE FUNCTION list_incoming_files()
RETURNS SETOF text AS
$$
import os
return os.listdir('/tmp')
$$
LANGUAGE 'plpython2u' VOLATILE SECURITY DEFINER;
\end{pyglist}


Para ejecutarlo:\\

\begin{pyglist}
SELECT * FROM list_incoming_files();
\end{pyglist}

Ejemplo de acceso a data desde plpython mediante la librería plpy.\\

\begin{pyglist}[language=python]
CREATE FUNCTION try_adding_joe() RETURNS text AS $$
    try:
        plpy.execute("INSERT INTO users(username) VALUES ('joe')")
    except plpy.SPIError:
        return "something went wrong"
    else:
        return "Joe added"
$$ LANGUAGE plpythonu;
\end{pyglist}

\begin{pyglist}[language=python]
CREATE FUNCTION count_odd_fetch(batch_size integer) RETURNS integer AS $$
odd = 0
cursor = plpy.cursor("SELECT num FROM largetable")
while True:
    rows = cursor.fetch(batch_size)
    if not rows:
        break
    for row in rows:
        if row['num'] % 2:
            odd += 1
return odd
$$ LANGUAGE plpythonu;
\end{pyglist}



\subsection{Aspectos básicos de Python}

Python es un lenguaje de programación interpretado, dinámico y multiparadigma (POO, estructurado). Se caracteriza por tener una baja curva de aprendizaje, ser multiuso y permitir una programación ordenada y elegante.\\

Su visión está expresada en el Zen de Python:\\

\begin{quotation}
Bello es mejor que feo.\\
Explícito es mejor que implícito.\\
Simple es mejor que complejo.\\
\end{quotation}

Python se ha utilizado para todo tipo de desarrollos incluyendo Web, aplicaciones de escritorio, sistemas de información, utilitarios, etc.\\



\subsection{Ejemplo de Python}

\begin{pyglist}[language=python]
import smtplib
from email.MIMEText import MIMEText

fromaddr = 'pmunoz@gmail.com'
toaddrs  = 'pedro@simuder.com'
msg = 'Hola, este es un mensaje de prueba'

message = MIMEText(msg)
message['From'] = 'Example <' + fromaddr  + '>'
message['To'] = toaddrs
message['MIME-Version'] = '1.0'
message['Content-type'] = 'text/HTML'
message['Subject'] = 'Asunto'

# Credentials (if needed)  
usuario = 'usuario'
password = 'password'

# The actual mail send  
server = smtplib.SMTP('smtp.gmail.com:587')
server.starttls()
server.login(usuario,password)
server.sendmail(fromaddr, toaddrs, message.as_string())
server.quit()
\end{pyglist}


Para empezar a experimentar con Python se puede utilizar el intérprete interactivo desde la consola:\\

\emph{python}\\


\subsection{Tipos básicos}

Los tipos básicos son en esencia tres: números, cadenas y valores booleanos.

\begin{itemize}
\item Números: 5 (entero), 4.7 (flotante), 4 + 6j (complejo).
\item Cadenas: "Gobierno regional".
\item Booleanos: True y False
\end{itemize}

Los comentarios se escriben utilizando el símbolo \#:\\

\begin{pyglist} [language=python]
\# Este es un comentario
a = "Esta es una linea de programa"
\end{pyglist}

Las variables se asignan de forma dinámica y dependen del tipo de dato que se les asigna, por ejemplo, una variable puede ser entera, luego asignarsele un valor de coma flotante y por último una cadena y es válido:\\

\begin{pyglist} [language=python]
>>> a = 5
>>> a
5
>>> a = 6.4
>>> a
6.4
>>> a = "Chiclayo"
>>> a
'Chiclayo'
>>> 
\end{pyglist}

\subsubsection{Cadenas}

Existen tres tipo de cadenas: Las cadenas normales, las cadenas unicode y las cadenas ``crudas'' (raw en inglés).\\

$>>>$ a = ``ñ''  \\
$>>>$ print ``a''   \\
ñ               \\
$>>>$ b =``ñ''   \\
$>>>$ print b   \\
ñ               \\
$>>>$ a         \\
u'$\backslash$xf1'   \\
$>>>$ b              \\
'$\backslash$xc3$\backslash$xb1'  \\
$>>>$ c = r``$\backslash$n''        \\
$>>>$ c                           \\
'$\backslash$$\backslash$n'       \\
$>>>$ print c                     \\ 
$\backslash$n                     \\
$>>>$                             \\


\subsection{Operadores}

\begin{table}[h]
\centering
\begin{tabular}{|c|c|c|} \hline
Operador	             & Descripción & Ejemplo     \\ \hline
+         & Suma               & r = 3 + 2    \# r es 5   \\ \hline
-         & Resta              & r = 4 - 7    \# r es -3   \\ \hline
-         & Negación           & r = -7       \# r es -7   \\ \hline     
*         & Multiplicación     & r = 2 * 6    \# r es 12   \\ \hline
**        & Exponente          & r = 2 ** 6   \# r es 64    \\ \hline
/         & División           & r = 3.5 / 2  \# r es 1.75    \\ \hline
//        & División entera    & r = 3.5 // 2 \# r es 1.0    \\ \hline
\%         & Módulo             & r = 7\ % 2    \# r es 1   \\ \hline
\end{tabular}
\end{table}

Si en una operación utilizamos un número flotante y un entero el resultado será flotante.\\

\subsection{Colecciones}

Los tipos de colecciones de datos más utilizados en Python son: listas, tuplas y diccionarios.\\

\subsubsection{Listas}

Las listas son el caballo de batalla de Python. Son colecciones de datoa que pueden tener cualquier tipo de dato: números, cadenas, otras listas, objetos, etc.\\

Para crear una lista se agrupan los objetos en corchetes separados por comas:\\

\begin{pyglist}[language=python]
>>> l = [22, True, "una lista", [1, 2]]
>>> l[2]
'una lista'
>>> l[3][0]
1
>>> l[1] = 15
>>> l
[22, 15, 'una lista', [1, 2]]
>>> 
>>> l[0:2]
[22, 15]
>>> l[:2]
[22, 15]
>>> l.append("mee")
>>> l
[22, 15, 'una lista', [1, 2], 'mee']
\end{pyglist}

\subsubsection{Tuplas}

Las tuplas se parecen a las listas, excepto por dos puntos: Se definen de forma diferente y son inmutables.\\

\begin{pyglist}[language=python]
>>> tupla = (1,5,"Hola", True, "Hotel")
>>> tupla[0]
1
>>> tupla[0:2]
(1, 5)
>>> t = (1)
>>> type(t)
<type 'int'>
>>> t = (1,)
>>> type(t)
<type 'tuple'>
\end{pyglist}

A cambio de no ser modificables, las tuplas son más ligeras en memoria.

\subsubsection{Diccionarios}

Son colecciones que relacionan una llave y un valor. El primer valor es la clave y el segundo el valor asociado a la clave. Como clave se puede utilizar cualquier valor inmutable: números, cadenas, booleanos, tuplas. A un diccionario se accede solo por la llave, ya que no tienen orden interno.\\

\begin{pyglist}[language=python]
>>> d = {"Don Corleone": "Marlon Brando", "Michel Corleone": "Al Pacino"}
>>> d
{'Michel Corleone': 'Al Pacino', 'Don Corleone': 'Marlon Brando'}
>>> d["Don Corleone"]
'Marlon Brando'
\end{pyglist}

\subsection{Control de flujo}

El control de flujo son los condicionales y bucles:\\

\begin{pyglist}
x = 10
y = 50

if x > 10 and y > 60:
    print "x mayor que 10 e y mayor que 60"
elif x > 5 and y < 100:
    print "x mayor que 5 e y menor que 100"
else:
    print "ninguna de las anteriores"
\end{pyglist}

En Python no existe el \textbf{switch}.\\

\begin{pyglist}
i = 0

while i < 20:
    print str(i) + " es menos de 20"
    i = i + 1
\end{pyglist}


En Python, el bucle \textbf{for ... in} se utiliza como una forma genérica de iterar sobre una secuencia.\\

\begin{pyglist}
x = 5
for j in range(x):
    print "a " + str(j) + " le faltan " + str(x - j) + " para alcanzar a " + str(x)
    
l = [4, "Segundo", "Chiclayo", False, 6.8]

for item in l:
    print item
\end{pyglist}

\subsection{Funciones}

Una función es un fragmento de código con un nombre asociado que realiza una serie de tareas y devuelve un valor.\\

\begin{pyglist}
def primera_funcion(param1, param2):

    suma = param1 + param2
    return suma

print primera_funcion(5,4)
print "\n"
print primera_funcion("Ho","la")

def varios(param1, param2, *otros):
    """Funcion con numero variable de parametros:W
    """
    print param1
    print param2
    for val in otros:
        print val

varios(1,2)
varios(1,2,3,4)
varios(1,2,3,4,5)
\end{pyglist}


\subsection{Funciones Trigger}

PostgreSQL ofrece Triggers a nivel de registro y de tablas y vistas. PostgreSQL ofrece funciones especializadas en gestionar los triggers que actuan como cualquier otra función y tienen la misma estructura básica. Donde difieren es en el parámetro de entrada y en la salida. \\

Una función trigger nunca recibe un input aunque internamente tiene acceso a la data del trigger y puede modificarla. Su salida siempre es un tipo llamado \textbf{trigger}. Cada trigger debe tener asociada unafunción trigger y para tener múltiples triggers se deben crear múltiples funciones sobre el mismo evento las cuales se ejecutaran en orden alfabético. \\

Se puede usar cualquier lenguaje menos SQL para escribir triggers. El más utilizado es PL/pgSQL. Se siguen dos pasos: en primer lugar escribir la función y luego el trigger correspondiente. \\

Por ejemplo:\\

\begin{pyglist}
CREATE OR REPLACE FUNCTION orders_2004_01_update_trigger()
RETURNS TRIGGER AS $$
BEGIN
    IF ( NEW.orderdate != OLD.orderdate ) THEN
      DELETE FROM orders_2004_01
        WHERE OLD.orderid=orderid;
      INSERT INTO orders values(NEW.*);
    END IF;
    RETURN NULL;
END;
$$
LANGUAGE plpgsql;

CREATE TRIGGER update_orders_2004_01
    BEFORE UPDATE ON orders_2004_01
    FOR EACH ROW
    EXECUTE PROCEDURE orders_2004_01_update_trigger();
\end{pyglist}

\begin{itemize}
\item Lo primero que se hace es definir la función trigger que elimina un valor antiguo e inserta el valor nuevo del registro.
\item El trigger se ejecutará antes de que se actualize un registro de la tabla \textit{orders\_2004\_01}.
\end{itemize}



